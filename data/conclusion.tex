% !Mode:: "TeX:UTF-8"
\begin{conclusion}

本项目提出了一种复杂环境下的高效自适应二维码编解码方案,可以有效的抵抗画面发生的一定程度的形变、色彩偏移,并且保证传输内容的高度可靠性。整套解决方案没有高级计算机图形库与复杂神经网络的参与,并且编解码效率非常高。

基于复杂环境下的高效自适应二维码编解码技术,本项目完成了一套基于自定义二维码与图像识别技术的单向网闸,研究了一套基于离散二维码的高可靠、可纠错的文件传输协议,并基于此协议实现了基于二维码的文件单向传输信道。

在传输的过程中,针对传输的内容进行压缩,调节二维码版本,对写入过程进行优化,调节纠错参数,实现了没有反向确认信道情况下保证99.999\%可靠性的,在传输工况下可能传输的文字内容的情况下(数据经过压缩)传输速率达到24Mbps的通信信道。

设计了一套适用于本系统的,与标准机架兼容的硬件设备,并且能与软件协同工作,工作状态符合预期。

在中国核工业集团福清核电站部署了包括硬件、软件在内的整套传输系统,正在进行长期稳定性测试。

\end{conclusion}