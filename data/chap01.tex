% !Mode:: "TeX:UTF-8"
\mychapter{绪论}
\label{cha:intro}

本章主要描述了单向网闸发展的背景,分析了相关的现状,进而提出本文所要研究的基于图像识别的二维码单向网闸的内容以及研究目标。

\section{单向网闸发展的背景}

在政府部门建设内部网络的工程中,安全与保密问题始终是建设的重点内容,这是因为内部网络中传输的信息常常是涉密信息或内部信息\cite{李结松2002办公网络安全策略研究及技术实现,马剑沁0信息化背景下的保密管理研究}。国家为此出台了一系列条例,包括:《关于加强政府上网信息保密管理的通知》、《计算机信息系统国际联网保密管理规定》、《关于进一步推进全国政府系统办公自动化建设和应用工作的通知》、《关于转发国家信息化 小组〈关于我国电子政务建设指导意见〉通知》等,明确规定:涉及国家秘密的计算机信息系统,不得直接或间接地与互联网或其它公共信息网络联接,必须实行物理隔离,以确保国家秘密的安全\cite{彭小利2010专网共享公网信息的研究}。

在2019年实施的《信息安全技术 网络安全等级保护基本要求》中,亦明确规定:工业控制系统与企业其他系统之间应划分为两个不同的网络区域,区域间应当采用技术手段进行隔离;工控系统的内部网络也应该当根据业务特点划分为不同的网络安全域,安全域之间同样应采用技术手段进行隔离;在网络边界不能直接进行连接,而是应该通过通信协议转换或是通信协议隔离才能进行数据交换。在《中华人民共和国网络安全法》,《国家网络安全事件应急预案》等国家安全标准同样含有相似的要求\cite{刘文2019等保}。

\section{单向网闸的研究现状}

单向网闸作为工控网络边界控制的重要组成部分,一直是网络安全领域的关注对象。下面具体阐述单向网闸的研究现状与存在的问题。

\subsection{单向网闸国内外研究现状}
大多数传统的单向传输系统使用一个网闸来隔离两个独立的主机系统或者网络。其基本原理是阻止网络之间的协议连接,将数据包分解以及重组为静态的无协议数据,并对静态数据进行安全检查,包括协议检查以及内容扫描;经过确认后的安全数据被允许流入内部单元。\cite{万月亮2010基于光闸的单向传输系统可靠性研究}传统网闸是一种信息安全设备,它使用具有控制功能的固态读/写介质来连接两个独立的主机或者是主机系统。单向网闸连接的两个独立的主机系统或者网络没有直接通信的物理连接、逻辑连接、信息传输控制、协议,没有按照通用协议转发的信息包,只有元数据文件,只存在无协议的数据摆渡,只有对固态存储介质的 "读 "和 "写"。\cite{范毅2005基于}
具体的实现技术有下面两种:

1、数据泵技术(Data Pump): 为了实现数据从低级数据库到高级数据库的可靠复制,Myong H. Kang等人提出了pump技术,即 "安全存储和转发技术"。该方法通过反向确认的方法限制数据从内部向外部传输,只允许数据从外部向内部单向流动。数据泵技术以双向的通信为基础,逻辑的限制数据只在一个方向传输,而在相反的方向只传输控制信息,如确认接收、错误控制、流量控制等\cite{杨翰文0基于}。该通信协议只允许数据在一个方向上传输,但控制信息在两个方向上都可以流动;它也可以理解为半双工控制信道和单工信息信道的结合。因此,数据泵技术的单向网闸相对容易实现,并可与既定的通信协议一起使用。虽然数据泵技术中的数据是单向的,但协议控制信息是双向传递的,如果协议本身存在缺口,就有可能利用协议中的缺口实现反向数据传输。

2、数据二极管技术(Data Diode):将反向控制协议也取消,采用 "盲发 "的方式,即一方只发送,另一方只接收。在传输过程中数据有无错误,数据是否完整都不在信道层面处理,在反向上没有数据信道也没有控制信道,完全是单盲状态。也可以理解为单工通信信道,所以也被称为信息流的单向技术。\cite{杨翰文0基于}

数据二极管技术的产品化,国外已经趋于成熟,比较出名的有美国Owl公司,荷兰Fox-IT公司,澳大利亚Tenix公司,美国HP公司\cite{oh2015development}。国内的单向网闸产品还处于起步阶段,有产品推出的有中铁信安公司与国保金泰公司。基于图像识别的异构单向网闸在世界范围内的发展尚不充分。



\subsection{研究难点}

由于数据传输需要较高的带宽,要求项目必须拥有很高的二维码编码与解码能力。又由于数据传输过程中,先被投射到显示屏后又被摄像头采集,中间必然会产生变形与失真,程序必须很好的应对这一些问题。

由于没有物理连接,摄像头与显示屏、编码器与解码器无法通过传统流程通信。又由于摄像头与显示屏工作在不同时钟,并且刷新时间、刷新频率都不一致,使得同步时钟操作实际不可能,因此整套程序必需在完全异步情况下工作,并且不能发生大量丢包。

单个二维码的容量有限,大文件的传输需要将信息散布在多个二维码中。对于数据需要进行拆分-传输-再组装的过程,需要保证最后的传输数据不缺不重不错以及顺序正确。

实际连接在网络中时,需要处理的不是文件而是TCP、UDP或其他数据包,但是系统只能进行文件的传输,需要保留传输的所有状态进行协议转换,涉及对实际传输协议的解析。

由于项目要求很高的传输速率,以及项目采用平台的特性,高级别的计算机视觉库(比如OpenCV)以及复杂神经网络推理(3层以上的CNN)都不能被使用。

\section{本文的研究目标以及研究内容}

本文基于单向网闸的研究现状,通过实验得出单向网闸传输速率、可靠性的影响因素,实现各项关键模块,并最终构建一套基于二维码的光学单向传输系统,实现数据的无差错传输。

\subsection{研究目标}

总体目标:基于图像识别技术为基础,开发一套软硬件系统,实现物理隔离,使网络边界从物理原理上就不可能产生信息的泄漏。

本项目基于图像识别技术的数据传输,实现工控边界数据传输单向性,以及传输链路真正的物理隔离,杜绝物理回包;本方案采用基于图像识别的单向数据传输技术。在实际应用中,编码端和解码端之间不存在实物介质连接,作为信息载体介质的物质是光。基于显示器和摄像机的硬件架构为完全单向的链路。在这种结构中,解码端不具备将数据通过某种介质发射到编码端的能力,编码端也不具备从解码端接收数据的能力。具体来说,显示屏只能展示图片,而不可能接受任何信息,因为其没有任何信号接收器;另一方面,摄像头只能接收信息而不可能发送信息,因为其从物理上来说就没有相关的硬件能做到这一点。\cite{杨劲锋0基于连续视频图像捕获的二维条码解码技术研究与应用}

项目期望构造基于二维码进行图像识别的数据传输,实现数据到图像再到数据的转换。二维条码是以一定的几何图形按一定规则分布在平面上,依照黑白二进制的方式记录数据的符号信息;在编码编制上采用比特流的概念,以几何图形对应二进制,以文字值代表信息,由图像输入设备或光电扫描设备自动读取,实现自动信息处理;具有一定的数据校验功能。经过编码的二维码含有数据,对于二维码进行时序的排列,在屏幕上播放,便可以构造一条基于二维码的数据传输信道。\cite{钱军2010二维条形码在机关公文管理中的应用}

\subsection{研究内容}

本项目的研究内容主要包括以下几个方面: 第一,针对单向通信功能,在明确系统工作流程的基础上,设计适用于本系统的传输协议,对不同类型数据包的格式进行定义,对发送端分块编码与二维码展示、接收端拍摄图像与接收数据包等关键功能模块进行设计并实现。 第二,为了提高系统可靠性,对系统中影响丢包率与解码成功率的因素进行分析,并在传输中利用技术手段提高系统容错能力。第三,为了实现信息的稳定、快速传输,在明确系统关键参数的基础上,分析各关键参数之间的影响关系,调整编码的冗余度,在兼顾丢包率和信息传输速率的条件下,实现较高速率、高可靠性的通信。




